% Options for packages loaded elsewhere
\PassOptionsToPackage{unicode}{hyperref}
\PassOptionsToPackage{hyphens}{url}
%
\documentclass[
]{article}
\usepackage{lmodern}
\usepackage{amssymb,amsmath}
\usepackage{ifxetex,ifluatex}
\ifnum 0\ifxetex 1\fi\ifluatex 1\fi=0 % if pdftex
  \usepackage[T1]{fontenc}
  \usepackage[utf8]{inputenc}
  \usepackage{textcomp} % provide euro and other symbols
\else % if luatex or xetex
  \usepackage{unicode-math}
  \defaultfontfeatures{Scale=MatchLowercase}
  \defaultfontfeatures[\rmfamily]{Ligatures=TeX,Scale=1}
\fi
% Use upquote if available, for straight quotes in verbatim environments
\IfFileExists{upquote.sty}{\usepackage{upquote}}{}
\IfFileExists{microtype.sty}{% use microtype if available
  \usepackage[]{microtype}
  \UseMicrotypeSet[protrusion]{basicmath} % disable protrusion for tt fonts
}{}
\makeatletter
\@ifundefined{KOMAClassName}{% if non-KOMA class
  \IfFileExists{parskip.sty}{%
    \usepackage{parskip}
  }{% else
    \setlength{\parindent}{0pt}
    \setlength{\parskip}{6pt plus 2pt minus 1pt}}
}{% if KOMA class
  \KOMAoptions{parskip=half}}
\makeatother
\usepackage{xcolor}
\IfFileExists{xurl.sty}{\usepackage{xurl}}{} % add URL line breaks if available
\IfFileExists{bookmark.sty}{\usepackage{bookmark}}{\usepackage{hyperref}}
\hypersetup{
  pdftitle={Homework 1},
  pdfauthor={Pawel Linek},
  hidelinks,
  pdfcreator={LaTeX via pandoc}}
\urlstyle{same} % disable monospaced font for URLs
\usepackage[margin=1in]{geometry}
\usepackage{graphicx,grffile}
\makeatletter
\def\maxwidth{\ifdim\Gin@nat@width>\linewidth\linewidth\else\Gin@nat@width\fi}
\def\maxheight{\ifdim\Gin@nat@height>\textheight\textheight\else\Gin@nat@height\fi}
\makeatother
% Scale images if necessary, so that they will not overflow the page
% margins by default, and it is still possible to overwrite the defaults
% using explicit options in \includegraphics[width, height, ...]{}
\setkeys{Gin}{width=\maxwidth,height=\maxheight,keepaspectratio}
% Set default figure placement to htbp
\makeatletter
\def\fps@figure{htbp}
\makeatother
\setlength{\emergencystretch}{3em} % prevent overfull lines
\providecommand{\tightlist}{%
  \setlength{\itemsep}{0pt}\setlength{\parskip}{0pt}}
\setcounter{secnumdepth}{-\maxdimen} % remove section numbering

\title{Homework 1}
\author{Pawel Linek}
\date{2/23/2020}

\begin{document}
\maketitle

\hypertarget{data}{%
\subsection{Data}\label{data}}

The data we were provided was data on student-teacher ratios from
various countries over time in various school levels. To see the data
check out this link: \url{https://tinyurl.com/student-teacher}.

We were also asked to compare some of the data to data in
\textbf{gapminder}, a built in library in R which looks at various
country data such as gdp and population.

\hypertarget{layout-of-analysis}{%
\subsection{Layout of Analysis}\label{layout-of-analysis}}

My analysis of the data consists of:

\begin{itemize}
\tightlist
\item
  \emph{Compare Geographic location and student-teacher ratio}
\item
  \emph{Compare population of continents to student-teacher ratio}
\item
  \emph{Compare GDP per Capita to student-teacher ratio}
\end{itemize}

I will be attempting to find reasons as to why some countries have
certain student-ratios.

\hypertarget{analysis}{%
\subsection{Analysis}\label{analysis}}

\hypertarget{geographic-location-vs.-student-teacher-ratios}{%
\subsubsection{Geographic Location vs.~Student-Teacher
Ratios}\label{geographic-location-vs.-student-teacher-ratios}}

My first instinct was to target geographic location and find the mean
student ratio per continent over time. I wanted to see whether there was
an increasing, decreasing or neutral trend of the student ratio over
time per continent. Below is my plot to depict this:

\includegraphics{Homework-1_files/figure-latex/unnamed-chunk-2-1.pdf}

A couple points we can deduce from the graph include:

\begin{itemize}
\item
  \emph{As can be seen above, Africa as a continent seems to have the
  largest student-teacher ratio with a negative trend. Meaning that over
  time, more teachers are being hired, less students are attending
  school (or a combination of both). We can imagine this is due to the
  large population size and lower educational standards. }
\item
  \emph{Population as a factor seems to be supported with Asia being the
  continent with the second largest student-teacher ratio, except in
  their case, the ratio seems to be increasing.}
\item
  \emph{Europe by far has the smalled student-teacher ratio of any
  continent, having approximately a \textbf{3x smaller} ratio than
  Africa and around 2.5 smaller than the next smallest continent.}
\end{itemize}

\hypertarget{population-vs.-student-teacher-ratios}{%
\subsubsection{Population vs.~Student-Teacher
Ratios}\label{population-vs.-student-teacher-ratios}}

The previous plot made me believe that population may have a role in the
student-teacher ratio of a country, and so I decided to drill-down on
this hypothesis.

I decided to take the countries with the smallest and the largest
population of each continent and compare their range of ratios overtime.
This helped me to see whether there were outliers in each continent, as
well as seeing how much data there seems to be per continent.

Below is my plot to depict the range:

\includegraphics{Homework-1_files/figure-latex/unnamed-chunk-4-1.pdf}

Looking at this plot shows us the range of the student-teacher ratios
focusing specifically on the most populous and least populous country in
each continent. Looking at Africa we see that Sao Tome and Principe's
student-teacher ratio is around 24 while being the least populous
country in Africa. Compare that to the least populous country in Europe
(Iceland), with student-teacher ratio equal to around 5, we notice that
population may not play the most significant role in student teacher
ratio.

Similarly if we compare the most populous country from each continent we
see that Ethiopia in Africa has a 41 student-teacher ratio, while China
in Asia has approximately a 24 student-teacher ratio.

\textbf{China's ratio is even less than Sao Tome and Principe's while
having a population significantly larger.}

\textbf{Population doesn't seem to be a significant independant variable
for explaining the student-teacher ratio of a country.}

Digging a little deeper into this, seeing how these extreme population
cases per continent are changing might give us insight as to how the
entire continent is trending, whether the outliers follow that trend or
whether they are minimizing it. Below is a plot representing this view:

\includegraphics{Homework-1_files/figure-latex/unnamed-chunk-5-1.pdf}

Viewing this graph I noticed that there was some missing data for the
European countries. There didn't seem to be any datapoints for the years
2016-2017 in general for this. Whether that is the case for the rest of
Europe I'm not sure but it may be a possiblity as to why European
countries have such low student-teacher ratios. Here we also notice that
Oceania only has one country as it's representative (New Zealand),
meaning it is both the min and the max for OCeania.

Looking at some of the trends of these countries, it looks like most
tend to follow their continents respective trendlines except for the
African Countries and Jamaica. While most other countries seemed to have
stagnated around the 2014 mark, we see Ethiopia's student-teacher ratio
grew substantially from 2014-2016 giving the inclination that either a
bunch of teachers left the profession, or less students entered the
education system.

A quick google search led me to this article
\url{https://www.refworld.org/docid/55505d0115.html}, which was a paper
written in 2014 titled \emph{Education under Attack 2014 - Ethiopia}. I
didn't get too deep into the article but it may provide some insight
into those years in Ethiopia.

Sao Tome and Principe followed a similar trend to Ethioipia, while
Jamaica recently in 2016-2017 seemed to have increased it's ratio.

\hypertarget{gdp-per-capita-vs.-student-teacher-ratios}{%
\subsubsection{GDP Per Capita vs.~Student-Teacher
Ratios}\label{gdp-per-capita-vs.-student-teacher-ratios}}

A hypothesis I had from the start was that GDP Per Capita, \emph{the
more affluent a country is, the smaller their student-teacher ratio
would be}.

I decided to investigate this hypothesis and plot every country's GDP
Per Capita vs.~Mean student-teacher ratio of that country splitting the
data up per continent. Below are my output graphs:

\includegraphics{Homework-1_files/figure-latex/unnamed-chunk-7-1.pdf}

This facet-wrapped plot revealed that there seems to be a large
correlation between gdp per capita and student ratio across every
continent (except Oceania as only 1 country exists in our dataset from
that region).

\hypertarget{takeaways}{%
\paragraph{Takeaways}\label{takeaways}}

\begin{itemize}
\tightlist
\item
  The more affluent the country is the lower their student-teacher seems
  to be, supported by a linear regression line.
\item
  GDP per Capita isn't a 100\% correlated variable as many countries
  have the same GDP per capita how varying student-teacher ratio. This
  difference can be a result of other compounding variables such as
  population, type of schooling that it is, etc.
\item
  European countries seemed to have the lowest student-teacher ratios
  while being a region much more affluent than other continents. This
  may result in better pay and benefits for teachers, making teachers
  much more readily available, as opposed to Afriacn countries which may
  have other factors dissuading people pursuing the career.
\end{itemize}

\hypertarget{conclusion}{%
\subsubsection{Conclusion}\label{conclusion}}

OVerall it seems that the main factors from the data affecting
student-teacher ratios (in order) are: * GDP Per Capita * Region *
Population

These are undoubtedly not the only variables affecting student-teacher
ratios. Factors such as culture, benefits, regulation and other things
not represented in the data surely played a huge impact on
student-teacher ratios. From what we can tell with this data, more
affluent countries, in Europe, Asia and the America's seem to have lower
student-teacher ratios as opposed to poorer countries.

\end{document}
